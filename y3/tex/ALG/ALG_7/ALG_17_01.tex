%This is my super simple Real Analysis Homework template

\documentclass{article}
% \usepackage[utf8]{inputenc}
\usepackage[polish]{babel}
\usepackage[]{amsthm} %lets us use \begin{proof}
\usepackage[]{amssymb} %gives us the character \varnothing
\usepackage{graphicx}
\usepackage{amsmath}
\usepackage[T1]{fontenc}
\usepackage{lmodern}
\usepackage{mathtools}
\usepackage[shortlabels]{enumitem}
\usepackage{tikz}
\usetikzlibrary{shapes,shapes.multipart}
\usetikzlibrary{arrows,automata,positioning}
\usepackage{algorithm,algorithmic}



\title{Algebra, PD 7 }
\author{Heorhii Lopatin}
\date{\today}

\begin{document}
\maketitle %This command prints the title based on information  
\section*{Zadanie 1}
{
  Rozłóż elementy $a=15-15i, b=7-i$ na czynniki nierozkładalne w pierścieniu $\mathbb{Z}[i]$. Użyj tych rozkładów do wyznaczenia $NWD(a,b)$ w $\mathbb{Z}[i]$.
\subsubsection*{Rozwiązanie}
{
  Korzystając z wiedzy z wykładu, liczby nierozkładalne w $\mathbb{Z}[i]$ to liczby pierwsze w $\mathbb{Z}$ postaci $4k+3$ i liczby postaci $a+bi$, że $a^2+b^2$ jest liczbą pierwszą.\\
  $a=15(1-i)=3\cdot5(1-i)=3(2+i)(2-i)(1-i)$ - każdy z czynników jest nierozkładalny.\\
  $b=7-i=(2-i)(3+i)=(2-i)(1+2i)(1-i)$ każdy z czynnkiów jest nierozkładalny.\\ 
  Więc $NWD(a,b)=(2-i)(1-i)$.
}

\section*{Zadanie 2}
{
  Użyj algorytmu Euklidesa w $\mathbb{Q}[x]$ do wyznaczenia $NWD(f,g)$, gdzie $f = x^4 + x^3+x^2+2x+1,g=x^3+2x^2+2x+1$. Wyznacz wielomiany $u,v\in\mathbb{Q}[x]$ takie, że $fu+gv= NWD(f,g)$.
}
\subsubsection*{Rozwiązanie}
{


}
\section*{Zadanie 3}
{
  
%Zadanie 3. Niech R1,R2 be,da, pier ́scieniami. Wykaz ̇, z ̇e kaz ̇dy ideal I w iloczynie prostym R1×R2 jestpostaciI1×I2,gdzieIj jestpewnymidealemwRj,dlaj=1,2.
Niech $R_1,R_2$ będą pierścieniami. Wykaż, że każdy ideał $I$ w iloczynie prostym $R_1\times R_2$ jest postaci $I_1\times I_2$, gdzie $I_j$ jest pewnym idealem w $R_j$, dla $j=1,2$.
}

\subsubsection*{Rozwiązanie}
{
Pokażmy równoważnie, że $I=e_1I+e_2I(\because (a,b) = e_1a+e_2b)$, gdzie $e_1=(1,0), e_2=(0,1)$.\\
Skoro $I$ jest zamknięty na mnożenie i jest grupą z dodawaniem, $e_iI\subseteq I \implies e_1I+e_2I \subseteq I$\\
Z drugiej strony, $\forall x \in I \; x = e_RI=x=(e_1+e_2)x\in e_1I+e_2I \implies I \subseteq e_1I+e_2I$\\ 
Więc, $I=e_1I+e_2I$\\


}

 

}
\end{document}
